\documentclass[12pt]{article}
\title{ECE 141 Homework 4}
\usepackage{subcaption}
\author{Lawrence Liu}
\usepackage{graphicx}
\usepackage{amsmath}
\usepackage{placeins}
\newcommand{\Laplace}{\mathscr{L}}
\setlength{\parskip}{\baselineskip}%
\setlength{\parindent}{0pt}%
\usepackage{xcolor}
\usepackage{listings}
\definecolor{backcolour}{rgb}{0.95,0.95,0.92}
\usepackage{amssymb}
\lstdefinestyle{mystyle}{
    backgroundcolor=\color{backcolour}}
\lstset{style=mystyle}

\begin{document}
\maketitle
\section*{Problem 1}
We have that $$\beta=\arctan(\frac{l_r}{l_r+l_f}\tan(u))$$
therefore
$$\tan(\beta)=\frac{l_r}{l_r+l_f}\tan(u)$$
therefore since the range of $\tan$ is $-\infty$ to $\infty$ for any $\beta$ we can find a $u$ that satisfies the equation.

\section*{Problem 2}
We have that
$$\frac{d}{dt}y=v\sin(\psi+\beta)$$
$$\frac{d}{dt}\psi=\frac{v}{l_R}\sin(\beta)$$
$$\beta=\arctan(\frac{l_r}{l_r+l_f}\tan(u))$$
Linearizing around $\psi=0$ $\beta=0$, we have
$$\frac{d}{dt}y=v(\psi+\beta)$$
$$\frac{d}{dt}\psi=\frac{v}{l_R}\beta$$
therefore taking the laplace transform we have
$$sY=v(\psi+\beta)$$
$$s\psi=\frac{v}{l_r}\beta$$
Therefore we get
$$sY=v(\frac{v}{l_r s}+1)\beta$$
Therefore the transfer function is
$$\frac{Y(s)}{\beta}=\frac{v(v+l_r s)}{l_r s^2}$$
So now with a controller $C(s)$ we have that the transfer 
\end{document}